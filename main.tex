\documentclass{es-template}

\usepackage{hyperref}
\hypersetup{colorlinks=true,linkcolor=black,citecolor=black,urlcolor=blue}
\usepackage{fontspec} % 确保可以用系统字体



% Title & Authors — sample filled from your DOCX to demonstrate layout
\title{The Separation of Oil from Oil-Contaminated Soils Using Thermal Treatment}

\title{The Separation of Oil from Oil-Contaminated Soils Using Thermal Treatment}

\authors{Yerbol Tileuberdi\textsuperscript{1,2,}\corr, Mukhtar Yeleuov\textsuperscript{2,3}, Zhazira Mukataeva\textsuperscript{1}, Yerlan Doszhanov\textsuperscript{2,}\corr}

\affils{
\textsuperscript{1}Abai Kazakh National Pedagogical University, 13 Dostyk Ave., Almaty, 050010, Kazakhstan\\
\textsuperscript{2}Institute of Combustion Problems, 172 Bogenbai Batyr Str., Almaty, 050012, Kazakhstan\\
\textsuperscript{3}Engineering and Science Hub, 38 Tulebaev Str., Almaty, 050004, Kazakhstan\\
\corr~Email:~\textcolor{blue}{er.tileuberdi@gmail.com} \textup{(Y. Tileuberdi)};~Email:~\textcolor{blue}{doszhanov\_yerlan@mail.ru} \textup{(Y. Doszhanov)}
}
% \affils{
%   \textsuperscript{1}Abai Kazakh National Pedagogical University, 13 Dostyk Ave., Almaty, 050010, Kazakhstan\\
%   \textsuperscript{2}Institute of Combustion Problems, 172 Bogenbai Batyr Str., Almaty, 050012, Kazakhstan\\
%   \textsuperscript{3}Engineering and Science Hub, 38 Tulebaev Str., Almaty, 050004, Kazakhstan\\[0.5em] % 与上面段落间隔
%   \emailauthor{Alice (alice@example.com), Bob (bob@example.com)}
% }


\date{\today} % 或者其他日期


% \author{Yerbol Tileuberdi\textsuperscript{1,2}\corr, Mukhtar Yeleuov\textsuperscript{2,3}, Zhazira Mukataeva\textsuperscript{1}, Yerlan Doszhanov\textsuperscript{2}\corr}
% \date{%
% \textsuperscript{1}Abai Kazakh National Pedagogical University, 13 Dostyk Ave., Almaty, 050010, Kazakhstan\\
% \textsuperscript{2}Institute of Combustion Problems, 172 Bogenbai Batyr Str., Almaty, 050012, Kazakhstan\\
% \textsuperscript{3}Engineering and Science Hub, 38 Tulebaev Str., Almaty, 050004, Kazakhstan\\
% \corr~\email{er.tileuberdi@gmail.com}; \corr~\email{doszhanov\_yerlan@mail.ru}
% }

\date{} % 留空即可


\begin{document}
\maketitle

\begin{abstract}
In the paper, the oil separated from the oil-contaminated soil using thermal treatment, and the characteristics of the separated oil were studied. Thermal treatment was carried out under conditions of uniform heating from room temperature to 450C. The heating rate of the oil spilled on the soil was 15  per minute. The average duration of the thermal treatment process of the research object was 50 minutes. As the proportion of polluting oil in the soil sample obtained from the neutralization of oil spilled on the soil increases, an increase in the percentage of the separated oil (liquid product) is observed. The liquid products were subjected to physical and chemical analysis. The resulting liquid product had a water content of 7.6
\end{abstract}

\begin{keywords}
Separation; Oil-contaminated soil; Thermal treatment; Oil pollution.
\end{keywords}

\begin{tocmodule}[figs/000001.jpg]{This paper presents a novel method for oil separation...}
\end{tocmodule}

\newpage

\section{Introduction}
A major source of energy, oil (or petroleum) powers industries, transportation, and the production of electricity\cite{vidonish2016thermal}. Oil pollution is thus a complex and multifaceted problem requiring an integrated approach to research and the development of effective remediation techniques.

A major source of energy, oil (or petroleum) powers industries, transportation, and the production of electricity\cite{vidonish2016thermal}. Oil pollution is thus a complex and multifaceted problem requiring an integrated approach to research and the development of effective remediation techniques.

\section{Materials and methods}
\subsection{Materials and sample preparation}
Briefly describe samples and preparation steps here.

\subsection{Experimental setup for thermal treatment}
Describe the reactor, heating program, and measurement workflow.

\section{Results and discussion}
Summarize the main quantitative results. Refer to \figref{fig:flow} and \tabref{tab:example}.

\begin{figure}[ht]
  \centering
  \includegraphics[width=0.85\linewidth]{figs/000001.jpg}
  \caption{Schematic diagram of the thermal treatment process for oil-contaminated soil.}
  \label{fig:flow}
\end{figure}

\begin{table}[ht]
  \centering
  \caption{Example table showing gas, liquid, and solid yields.}
  \label{tab:example}
  \begin{tabular}{lccc}
    \toprule
    Sample & Gaseous (\%) & Liquid (\%) & Solid (\%)\\
    \midrule
    OCS-10 & 4.10 & 9.01 & 86.89\\
    OCS-30 & 5.42 & 23.15 & 71.43\\
    \bottomrule
  \end{tabular}
\end{table}

\section{Conclusion}
Conclude with key findings and implications.

\section*{Acknowledgments}
(Optional) Acknowledge funding and contributions.

\section*{Conflict of Interest}
There is no conflict of interest.

% \bibliographystyle{unsrtnat}
% \bibliography{references}
\printbibliography

\end{document}
